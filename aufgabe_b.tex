\subsection{Zwei Algorithmen}
Ziel dieser Aufgabe ist es, die äquivalenz von zwei Algorithmen zu zeigen.
Der Erste entspricht dem Algorithmus im Skriptum. 
\begin{algorithm}
\caption{Veriante aus dem Skriptum}
\label{av1}
\begin{algorithmic}[1]
\FOR{k=1,\dots,n}
    \STATE $L_{kk}\gets \sqrt{(A_{kk}-\sum_{j=1}^{k-1}L_{kj}^2)}$ \label{av1:diag}
    \FOR{i=k+1,\dots,n}
        \STATE $L_{ik}\gets (A_{ik}-\sum_{j=1}^{k-1}L_{ij}L_{kj})/L_{kk}$ \label{av1:elem}
    \ENDFOR
\ENDFOR
\end{algorithmic}
\end{algorithm}
Der zweite Algorithmus ist eine leichte Variation davon. Im Gegensatz zum
ursprünglichen Algorithmus arbeitet er direkt auf der Eingangsmatrix und
geht auch stärker Spaltenweise vor. Er wird in \ref{Aufgabe_c} implementiert.
\begin{algorithm}
\caption{Modifizierte Version}
\label{av2}
\begin{algorithmic}[1]
\FOR{k=1,\dots,n}
    \STATE $A_{kk}\gets \sqrt{A_{kk}}$ \label{av2:sqrt}
    \FOR{i=k+1,\dots,n} \label{av2:divs}
        \STATE $A_{ik}\gets A_{ik}/A_{kk}$ \label{av2:div}
    \ENDFOR \label{av2:dive}

    \FOR{i=k+1,\dots,n} \label{av2:sums}
        \FOR{j=i,\dots,n}
            \STATE $A_{ji}\gets A_{ji}-A_{jk}A_{ik}$ \label{av2:sum}
        \ENDFOR
    \ENDFOR \label{av2:sume}
    \label{av2:suma}
\ENDFOR
\end{algorithmic}
\end{algorithm}

\newcommand{\linespanK}{Zeile~\ref{av2:sqrt} bis \ref{av2:dive} }
Es soll nun gezeigt werden, dass diese zwei Algorithmen äquivalent sind.

Wir sehen, dass in Algorithmus~\ref{av1} die Elemente der $L$ Matrix jeweils nur
einmal verändert werden. Bei den Diagonalelementen passiert das in Zeile~\ref{av1:diag},
für alle anderen Elemente passiert das in Zeile~\ref{av1:elem}. 

In Algorithmus~\ref{av2} ist die Situation komplizierter. Die Laufvariable $k$ wird auch 
hier verwendet um die Spalten durch zu zählen, allerdings gibt es innerhalb der
Schleife mehrere Aspekte zu beachten. \linespanK arbeiten
jeweils nur mit Elementen aus der $k$. Spalte. Interessant sind Zeile~\ref{av2:sums} bis
\ref{av2:sume}. Da $j\ge i>k$ und $i,j\le n$ wird in Zeile~\ref{av2:sum} jedes Element, 
das in den auf die $k$. Spalte folgenden Spalten in der unteren Dreiecksmatrix steht,
bearbeitet. Dabei wird von jedem dieser Element ein Produkt aus zwei Elementen der $k$.
Spalte subtrahiert. Daraus folgt, dass ab Zeile~\ref{av2:dive} alle Elemente in den
Spalten $\le k$ nicht mehr verändert werden.

Sei $L_{xy}, x\ge y$ ein Element aus dem Resultat von Algorithmus~\ref{av1}, wenn das
korrespondierende Element $A_{xy}$ aus Algorithmus~\ref{av2} den selben Wert hat, dann
sind die Algorithmen äquivalent. In Zeile~\ref{av2:sum} von Algorithmus~\ref{av2} 
wird immer 
\begin{equation*}
    A_{xy}\gets A_{xy}-A_{xk}A_{yk} 
\end{equation*}
berechnet. Da sich die Elemente in der $k$. Spalte nicht mehr verändern, können wir 
dies schreiben als:
\begin{align*}
    A''_{xy}\gets& -A'_{x(y-1}A'_{y(y-1)}-A'_{x(y-2)}A'_{y(y-2)}-\dots-A'_{x1}A_{y1}+A'_{xy} \\
          &=A'_{xy}-\sum_{i=1}^{y-1}A'_{xi}A'_{yi}
\end{align*}
Wobei $A'_{mn}$ als der Wert an der Stelle $(m,n)$ nach Abschluss des Algorithmuses ist und 
$A''_{mn}$ der Wert an der Stelle $(m,n)$ bevor \linespanK 
auf diesen Wert ausgeführt werden, also der Wert dieses Elements in Zeile~\ref{av2:suma} hat
wenn $k=j-1$. Wenn wir nun $A'_{xy}$ erhalten wollen müssen wir \linespanK mit einbeziehen.
Dabei gibt es zwei Fälle. Der erste Fall tritt ein wenn $x=y$. In diesem Fall gilt: 
\begin{equation*}
    A'_{xx} \gets \sqrt{A'_{xx}-\sum_{i=1}^{y-1}A'_{xi}A'_{xi}} = 
                  \sqrt{A'_{xx}-\sum_{i=1}^{y-1}(A'_{xi})^2}
\end{equation*}
Im Fall $x>y$ gilt:
\begin{equation*}
    A'_{xy}\gets \frac{\left(A'_{xy}-\sum_{i=1}^{y-1}A'_{xi}A'_{yi}\right)}{A'_{xx}}
\end{equation*}
Was genau mit Zeile~\ref{av1:diag} bzw. \ref{av1:elem} aus Algorithmus~\ref{av1} 
übereinstimmt.
Wir sehen mit Induktion sofort, dass die $A'_{nm}$ aus Algorithmus~\ref{av2} mit den
$L_{nm}$ aus Algorithmus~\ref{av1} korrespondieren\footnote{wenn $m=1$ gilt: $L_{1n}=
A_{1n}=A'_{1n}$}.
